\documentclass[letterpaper,12pt]{article}
% preamble
\usepackage[utf8]{inputenc}
\usepackage{amsmath,amssymb}
\usepackage{physics}
\usepackage[bb=boondox,bbscaled=1.0]{mathalfa}
\usepackage[colorlinks=true, allcolors=blue]{hyperref}
\usepackage{graphicx}
\usepackage{xcolor}
% \usepackage{newtxtext,newtxmath}
%\usepackage{arev}
\renewcommand*\familydefault{\sfdefault} 
% \usepackage{sansmathfonts}
% \usepackage{newtxmath}
\usepackage[top=1in, bottom=1in, left=1in, right=1in,nomarginpar]{geometry}
\usepackage[most,many,breakable]{tcolorbox}
\usepackage{dsfont}
\usepackage{outlines}

% paragraphs
\setlength{\parskip}{1em}
\setlength{\parskip}{1em}

% header
\usepackage{fancyhdr}
\setlength{\headheight}{1.5em}
\fancyfoot{}
\fancyhead{}
\fancyhead[L]{Notes on \textbf{\@topic}}
\fancyhead[R]{\thepage}
%\fancyfoot[CE,RO]{\thepage}
%\fancyfoot[CE,CO]{\thepage}

% commands 
\renewcommand{\familydefault}{\sfdefault}
\newcommand{\mprop}[2]{\begin{Prop}{#1}{}#2\end{Prop}}
\newcommand{\mnote}[2]{\begin{Note}{#1}{}#2\end{Note}}
\newcommand{\mrem}[2]{\begin{Remark}{#1}{}#2\end{Remark}}
\newcommand{\mexam}[2]{\begin{Example}{#1}{}#2\end{Example}}

% definitions
\def\topic#1{\gdef\@topic{#1}}
\def\@topic{\@latex@warning@no@line{No \noexpand\topic given}}

% color boxes
\definecolor{mypropbg}{HTML}{f2fbfc}
\definecolor{mypropfr}{HTML}{191971}
\definecolor{mynotebg}{HTML}{f2fbfc}
\definecolor{mynotefr}{HTML}{191971}
\definecolor{myexamplebg}{HTML}{EFEFEF}
\definecolor{myexamplefr}{HTML}{999999}
\definecolor{myexampleti}{HTML}{444444}

\definecolor{wongOrange}{RGB}{230,159,000}
\definecolor{wongOrangeDark}{RGB}{175,120,000}
\definecolor{wongBlue}{RGB}{000,114,178}
\definecolor{wongVermillion}{RGB}{213,094,000}
\definecolor{wongGreen}{RGB}{000,158,115}

%================================
% NOTE
%================================

\tcbuselibrary{theorems,skins,hooks}
\newtcbtheorem[number within=section]{Note}{Note}
{%
	enhanced,
	breakable,
	colback = mynotebg,
	frame hidden,
	boxrule = 0sp,
	borderline west = {2pt}{0pt}{mynotefr},
	sharp corners,
	detach title,
	before upper = \tcbtitle\par\smallskip,
	coltitle = mynotefr,
	fonttitle = \bfseries\sffamily,
	description font = \mdseries,
	separator sign none,
	segmentation style={solid, mynotefr},
}
{th}

%================================
% REMARK
%================================

\tcbuselibrary{theorems,skins,hooks}
\newtcbtheorem[number within=section]{Remark}{Remark}
{%
	enhanced,
	breakable,
	colback = wongOrange!30!white,
	frame hidden,
	boxrule = 0sp,
	borderline west = {2pt}{0pt}{wongOrange!90!black},
	sharp corners,
	detach title,
	before upper = \tcbtitle\par\smallskip,
	coltitle = wongOrange!60!black,
	fonttitle = \bfseries\sffamily,
	description font = \mdseries,
	separator sign none,
	segmentation style={solid, wongOrange!90!black},
}
{th}

%================================
% EXAMPLE BOX
%================================

\newtcbtheorem[number within=section]{Example}{Example}
{%
	colback = myexamplebg
	,breakable
	,colframe = myexamplefr
	,coltitle = myexampleti
	,boxrule = 1pt
	,sharp corners
	,detach title
	,before upper=\tcbtitle\par\smallskip
	,fonttitle = \bfseries
	,description font = \mdseries
	,separator sign none
	,description delimiters parenthesis
}
{ex}


%================================
% PROPOSITION
%================================

\tcbuselibrary{theorems,skins,hooks}
\newtcbtheorem[number within=section]{Prop}{Proposition}
{%
	enhanced,
	breakable,
	colback = mypropbg,
	frame hidden,
	boxrule = 0sp,
	borderline west = {2pt}{0pt}{mypropfr},
	sharp corners,
	detach title,
	before upper = \tcbtitle\par\smallskip,
	coltitle = mypropfr,
	fonttitle = \bfseries\sffamily,
	description font = \mdseries,
	separator sign none,
	segmentation style={solid, mypropfr},
}
{th}

\tcbuselibrary{theorems,skins,hooks}
\newtcbtheorem[number within=chapter]{prop}{Proposition}
{%
	enhanced,
	breakable,
	colback = mypropbg,
	frame hidden,
	boxrule = 0sp,
	borderline west = {2pt}{0pt}{mypropfr},
	sharp corners,
	detach title,
	before upper = \tcbtitle\par\smallskip,
	coltitle = mypropfr,
	fonttitle = \bfseries\sffamily,
	description font = \mdseries,
	separator sign none,
	segmentation style={solid, mypropfr},
}
{th}


% math symbols
\newcommand*{\imag}{\mathrm{i}}

% math letters
%---------------------------------------
% BlackBoard Math Fonts :-
%---------------------------------------

%Captital Letters
\newcommand{\bbA}{\mathbb{A}}	\newcommand{\bbB}{\mathbb{B}}
\newcommand{\bbC}{\mathbb{C}}	\newcommand{\bbD}{\mathbb{D}}
\newcommand{\bbE}{\mathbb{E}}	\newcommand{\bbF}{\mathbb{F}}
\newcommand{\bbG}{\mathbb{G}}	\newcommand{\bbH}{\mathbb{H}}
\newcommand{\bbI}{\mathbb{I}}	\newcommand{\bbJ}{\mathbb{J}}
\newcommand{\bbK}{\mathbb{K}}	\newcommand{\bbL}{\mathbb{L}}
\newcommand{\bbM}{\mathbb{M}}	\newcommand{\bbN}{\mathbb{N}}
\newcommand{\bbO}{\mathbb{O}}	\newcommand{\bbP}{\mathbb{P}}
\newcommand{\bbQ}{\mathbb{Q}}	\newcommand{\bbR}{\mathbb{R}}
\newcommand{\bbS}{\mathbb{S}}	\newcommand{\bbT}{\mathbb{T}}
\newcommand{\bbU}{\mathbb{U}}	\newcommand{\bbV}{\mathbb{V}}
\newcommand{\bbW}{\mathbb{W}}	\newcommand{\bbX}{\mathbb{X}}
\newcommand{\bbY}{\mathbb{Y}}	\newcommand{\bbZ}{\mathbb{Z}}

%---------------------------------------
% MathCal Fonts :-
%---------------------------------------

%Captital Letters
\newcommand{\mcA}{\mathcal{A}}	\newcommand{\mcB}{\mathcal{B}}
\newcommand{\mcC}{\mathcal{C}}	\newcommand{\mcD}{\mathcal{D}}
\newcommand{\mcE}{\mathcal{E}}	\newcommand{\mcF}{\mathcal{F}}
\newcommand{\mcG}{\mathcal{G}}	\newcommand{\mcH}{\mathcal{H}}
\newcommand{\mcI}{\mathcal{I}}	\newcommand{\mcJ}{\mathcal{J}}
\newcommand{\mcK}{\mathcal{K}}	\newcommand{\mcL}{\mathcal{L}}
\newcommand{\mcM}{\mathcal{M}}	\newcommand{\mcN}{\mathcal{N}}
\newcommand{\mcO}{\mathcal{O}}	\newcommand{\mcP}{\mathcal{P}}
\newcommand{\mcQ}{\mathcal{Q}}	\newcommand{\mcR}{\mathcal{R}}
\newcommand{\mcS}{\mathcal{S}}	\newcommand{\mcT}{\mathcal{T}}
\newcommand{\mcU}{\mathcal{U}}	\newcommand{\mcV}{\mathcal{V}}
\newcommand{\mcW}{\mathcal{W}}	\newcommand{\mcX}{\mathcal{X}}
\newcommand{\mcY}{\mathcal{Y}}	\newcommand{\mcZ}{\mathcal{Z}}


%---------------------------------------
% Bold Math Fonts :-
%---------------------------------------

%Captital Letters
\newcommand{\bmA}{\boldsymbol{A}}	\newcommand{\bmB}{\boldsymbol{B}}
\newcommand{\bmC}{\boldsymbol{C}}	\newcommand{\bmD}{\boldsymbol{D}}
\newcommand{\bmE}{\boldsymbol{E}}	\newcommand{\bmF}{\boldsymbol{F}}
\newcommand{\bmG}{\boldsymbol{G}}	\newcommand{\bmH}{\boldsymbol{H}}
\newcommand{\bmI}{\boldsymbol{I}}	\newcommand{\bmJ}{\boldsymbol{J}}
\newcommand{\bmK}{\boldsymbol{K}}	\newcommand{\bmL}{\boldsymbol{L}}
\newcommand{\bmM}{\boldsymbol{M}}	\newcommand{\bmN}{\boldsymbol{N}}
\newcommand{\bmO}{\boldsymbol{O}}	\newcommand{\bmP}{\boldsymbol{P}}
\newcommand{\bmQ}{\boldsymbol{Q}}	\newcommand{\bmR}{\boldsymbol{R}}
\newcommand{\bmS}{\boldsymbol{S}}	\newcommand{\bmT}{\boldsymbol{T}}
\newcommand{\bmU}{\boldsymbol{U}}	\newcommand{\bmV}{\boldsymbol{V}}
\newcommand{\bmW}{\boldsymbol{W}}	\newcommand{\bmX}{\boldsymbol{X}}
\newcommand{\bmY}{\boldsymbol{Y}}	\newcommand{\bmZ}{\boldsymbol{Z}}
%Small Letters
\newcommand{\bma}{\boldsymbol{a}}	\newcommand{\bmb}{\boldsymbol{b}}
\newcommand{\bmc}{\boldsymbol{c}}	\newcommand{\bmd}{\boldsymbol{d}}
\newcommand{\bme}{\boldsymbol{e}}	\newcommand{\bmf}{\boldsymbol{f}}
\newcommand{\bmg}{\boldsymbol{g}}	\newcommand{\bmh}{\boldsymbol{h}}
\newcommand{\bmi}{\boldsymbol{i}}	\newcommand{\bmj}{\boldsymbol{j}}
\newcommand{\bmk}{\boldsymbol{k}}	\newcommand{\bml}{\boldsymbol{l}}
\newcommand{\bmm}{\boldsymbol{m}}	\newcommand{\bmn}{\boldsymbol{n}}
\newcommand{\bmo}{\boldsymbol{o}}	\newcommand{\bmp}{\boldsymbol{p}}
\newcommand{\bmq}{\boldsymbol{q}}	\newcommand{\bmr}{\boldsymbol{r}}
\newcommand{\bms}{\boldsymbol{s}}	\newcommand{\bmt}{\boldsymbol{t}}
\newcommand{\bmu}{\boldsymbol{u}}	\newcommand{\bmv}{\boldsymbol{v}}
\newcommand{\bmw}{\boldsymbol{w}}	\newcommand{\bmx}{\boldsymbol{x}}
\newcommand{\bmy}{\boldsymbol{y}}	\newcommand{\bmz}{\boldsymbol{z}}

% number sets
\newcommand{\RR}[1][]{\ensuremath{\ifstrempty{#1}{\mathbb{R}}{\mathbb{R}^{#1}}}}
\newcommand{\NN}[1][]{\ensuremath{\ifstrempty{#1}{\mathbb{N}}{\mathbb{N}^{#1}}}}
\newcommand{\ZZ}[1][]{\ensuremath{\ifstrempty{#1}{\mathbb{Z}}{\mathbb{Z}^{#1}}}}
\newcommand{\QQ}[1][]{\ensuremath{\ifstrempty{#1}{\mathbb{Q}}{\mathbb{Q}^{#1}}}}
\newcommand{\CC}[1][]{\ensuremath{\ifstrempty{#1}{\mathbb{C}}{\mathbb{C}^{#1}}}}
\newcommand{\PP}[1][]{\ensuremath{\ifstrempty{#1}{\mathbb{P}}{\mathbb{P}^{#1}}}}
\newcommand{\HH}[1][]{\ensuremath{\ifstrempty{#1}{\mathbb{H}}{\mathbb{H}^{#1}}}}
\newcommand{\FF}[1][]{\ensuremath{\ifstrempty{#1}{\mathbb{F}}{\mathbb{F}^{#1}}}}

\usepackage[numbers,sort]{natbib}
\usepackage{diagbox}
\bibliographystyle{unsrtnat}
\usepackage{algorithm}
\usepackage{algpseudocode}
\newcommand{\algorithmautorefname}{Algorithm}

% generics line 5: title, line 6: topic, line 7: date
\title{Numerical Methods}
\topic{Computational Physics}
\date{\today}



\author{Alberto Ruiz-Biestro}

\begin{document}
\maketitle
\tableofcontents
\thispagestyle{empty}

\clearpage\pagestyle{fancy}
\section{Finite Difference Methods}
A great resource for finite difference methods is the book by \citet{smith1985numerical}. Almost all of the information and methods can be traced back to that book.

\subsection{Wave Equation}
Hyperbolic equations are generally the description of vibrational problems. The (scalar) wave equation is the model hyperbolic equation,
\begin{equation}
  \pdv[2]{u}{t}-c^2\Delta u=f,
\end{equation}
where $c$ is the speed determined by the medium and $f$ is a source. Considering the homogeneous version of the previous equation. It will be useful to write it in operator form,
\begin{equation}
  D^2_t u - c^2\qty(D^2_x + D^2_y + D^2_z )u=0.
\end{equation}
We proceed as usual in discretizing the derivatives with a \emph{centered differences scheme} that has an error $\order{(h^2)^n}$. Our description of said equation will be at gridpoints $(p\,h,q_i\,k)$, with $h$ being the time step and $k$ the spatial step size\footnote{It need not be the same, but the discretization takes a more complicated form (non uniform grids)}. It will be useful to realize that a (cartesian) centered difference scheme has a general form of a sum of the value of the field at a set of gridpoints $\{x_i\}$ times a set of weights $\{w_i\}$,
\begin{equation}
  D_x u\Rightarrow \frac{1}{k_x}\qty(\dots +w_{i-1} u_{i-1}+ w_{i} u_{i} + w_{i+1} u_{i+1}+\dots),
\end{equation}
From \href{https://en.wikipedia.org/wiki/Finite_difference_coefficient}{Wikipedia} one gets the required weights for a second order derivative (see \autoref{tab:weights}).

\begin{table}[h!]
\centering
\caption{Weights for discretized $\pdv*[2]{x}$, with error $O(h^n)$ at grid index.}\label{tab:weights}
\begin{tabular}{|r|c|c|c|c|c|c|c|}
\hline
\diagbox{$n$}{idx}&$-3$&$-2$&$-1$&$0$&$1$&$2$&$3$\\\hline
2 &         &           & $1$       & $-2$       & $1$       &           &         \\\hline
4 &         & $-1/12$   & $16/12$   & $-30/12$   & $16/12$   & $-1/12$   &         \\\hline
6 & $2/180$ & $-27/180$ & $270/180$ & $-490/180$ & $270/180$ & $-27/180$ & $2/180$ \\\hline
\end{tabular}
\end{table}
\subsubsection{2D operator}
Discretizing a cartesian operator in a cartesian grid translates into summing the respective weights in distinct axes, since $\Delta=\sum_j\pdv*[2]{x_j}$. For example, for our two dimensional $D^2_x+D^2_y$, an approximation of $\order{h^2}$\footnote{Also known as a 5-point lattice} requires the sum of $[1,-2,1]$ and $[1,2,1]^T$. One finds the 2D version of the discretized wave equation to be,
\begin{equation}
  \frac{1}{h^2}\qty(u_{i,j}^{t-1}-2u_{i,j}^{t}+u_{i,j}^{t+1})-\frac{c^2}{h^2}\qty(\underbrace{\mqty[0& 1 & 0\\1&-4&1\\0&1&0]}_{\vb{D}^2}\vdot\mqty[ & u^t_{i,j+1} & \\ u^t_{i-1,j} & u^t_{i,j} & u^t_{i+1,j}\\& u^t_{i,j-1} &] )=0,
\end{equation}
where $\vdot$ implies the dot product. Alternatively, one calls the spatial discretization operator $\vb{D}^2$ and thus writes its product as $\vb{D}^2\vdot\vb{u}$. This expression already hints at the fact that one effectively \emph{convolves} the operator $\vb{D}^2$ with the lattice points, $\vb{D}^2\ast [u]^t$. This is what's done usually in image filters and graphics processing, and what is described in \autoref{alg:wave}.

Solving for $u^{t+1}$, one finds
\begin{equation}
  u_{i,j}^{t+1}=\alpha^2\qty(\vb{D}^2\vdot \vb{u})+2u_{i,j}^t - u_{i,j}^{t-1},
\end{equation}
where $\alpha\equiv ch/k$. One has the freedom of choosing each approximation's error, whether to use more coefficients for $D_t^2$ or for $\vb{D}^2$, or even take it to an $N-$dimensional routine by using a general tensor form of $\vb{D}^2$. One also notices that exchanging $u_{i,j}^{t+1}\leftrightarrow u_{i,j}^{t-1}$ has no effect whatsoever in the final expression.

As with all centered differences, one can kick-start the time-stepping through an explicit Euler scheme,
\[u_{i,j}^{t+1}=u_{i,j}^t+h(c^2\vb{D}^2\vdot \vb{u}),\]
which just implies having to state the initial conditions for $u_{i,j}^{(0)}$ and $u_{i,j}^{(-1)}$ (should one start at $t=0$).
 
Adding a source $f$ requires no further changes other than adding its value $f_{i,j,k}$ at time $t$. The algorithm for the time evolution of the wave equation in a domain $\Omega$, assuming a Dirichlet boundary conditions, and $V$ to be the initial perturbation, can be seen in \autoref{alg:wave}.

\begin{algorithm}
  \caption{FDTD for wave equation}\label{alg:wave}
  \begin{algorithmic}
    \State $u^{t-1}\gets 0$
    \State $u^{t}\gets V$
    \State $u^{t+1}\gets 0$
    \While{$N<N_{\rm max}$}
    \State $u^{t+1}=\alpha^2(\vb{D}^2\ast u^t)+2u^t-u^{t-1}$
    \State ${u}^{t+1}(\vb{r}\in\partial \Omega) = 0$\Comment{Enforce Dirichlet B.C.}
    \State $u^{t-1}\gets u^t$
    \State $u^{t}\gets u^{t+1}$ \Comment{Update fields}
    \EndWhile
  \end{algorithmic}
\end{algorithm}
\subsubsection{Image filters}
% Image filters have a long history of development, 
Using an image filtering library has the advantage of an easy implementation of periodic boundary conditions. In Julia, such function is given by \texttt{imfilter(im,ker,"circular")}. Furthermore, it allows generalization to $N$ dimensions, which makes it easy to read and work with. However, such operations can get expensive memory-wise, which is why it might also be recommended to use \texttt{for} loops. For Julia, this is no issue, since \texttt{for} loops should be as fast as Fortran code.

\bibliography{fdtd.bib}

\subsection{Ginzburg-Landau equation}
Ignoring some stability issues with the centered difference method outlined before, once we discretize the $\Delta$ operator, we can apply it to any other equation. Consider the equation
\begin{equation}
  \pdv{\phi}{t}=\phi-\abs{\phi}^2\phi+\Delta\phi,
\end{equation}
where we define $f(\phi)\equiv\phi-|\phi|^2\phi$ for convenience. Using a simple Euler approximation
\[\frac{1}{h}\qty(\phi_{i,j}^{t-1}-2\phi_{i,j}^t+\phi_{i,j}^{t+1})=f_{i,j}^t+\frac{1}{k^2}(\vb{D}^2\ast \phi)_{i,j}^t\]

\subsection{Crank-Nicholson method}
\section{Boundary-Wall method}
\subsection{BWM for Schr\"odinger's equation}
\subsection{BWM for other equations}

\end{document}
